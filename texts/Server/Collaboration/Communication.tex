\subsubsection{Communication}
For our collaboration we used five technologies to communicate and share information.

For sharing information we used Dropbox and Google Drive. \vspace{-4mm}
\begin{description}
	\item Dropbox \\
		We chose to use Dropbox for sharing non-document files. We chose Dropbox because it is a program all of us already had installed. We have used it before and it is a very easy tool for sharing files as long as multiple people are not to change a file at the same time.
	\item Google Drive \\
		For files that multiple people might edit at the same time, and documents in general. We chose Google Drive, which allows for multiple people to edit the same file at the same time. Since everyone already had a Google Drive account it was easy to set up.
		Google Drive was mainly used for sharing the API documentation for the web services.
\end{description}

When communicating with SMU we used Skype, Google Hangout and email.
\vspace{-4mm}
\begin{description}
	\item Skype \\
		For most of our meetings with SMU we used Skype for video calls. Skype was a great improvement compared to the video conference tool provided by ITU in regards to audio quality.
		Skype was also used as an instant message service when small problems needed to be solved and questions answered. \\
		The drawback of using Skype as a video conference tool is that when using the free plan (which we did), only two Skype accounts can be in a video conference at a time.
	\item Google Hangout \\
		If the group members of one of the groups were not able to meet at one place to conduct a video conference we used Google Hangout. With this tool up to 10 individual accounts can participate in one video conference at a time.
	\item Email \\
		To inform and communicate beside the video conferences the appointed liaisons emailed each other and passed the information on to the rest of their respective group members.
		The use of email eliminated problems that might arise by different and unknown accents.
\end{description}