\section{Introduction to collaboration with SMU}
In this project we had the amazing opportunity to collaborate with a group of students from the Singapore Management University.
\\It is a very relevant exercise for both parties as outsourcing is being used more and more.

We started our collaboration with a short video-chat, just to say hello and to try to give a good first impression. We held two video-chat meetings a week for the rest of the duration.

In the beginning, both parties made sure to formulate sentences very clearly and straight forward, trying not to say something, that could be perceived with a different meaning than intended. Communicating via text was way easier, and the accent was not a factor. \textbf{[TODO: Maybe write something like "since the accent...]}
\\For every meeting it became easier and more loose, and conversation became natural with their best English speaker.

From our group we chose Claus as our liaison-guy and they chose Lynette. They were the main responsible for communication between the groups and for keeping the other group well-informed.

We were very aware of the fact that our Singaporean group is of a different culture and that they might conceive something differently than we would. It was never a problem, though.

Singapore is 8 hours ahead of Denmark in the summertime. We usually held our meetings around noon our time, which is late afternoon Singaporean time. It was not hard to arrange meetings, but the time difference often made for long response times.

By the end of our collaboration each member from each group reflected upon the whole process. These can be found on our shared Wiki of which there is a copy in our Appendix.

Our shared Wiki page can be found in the Appendix and at
\\\url{wiki.smu.edu.sg/is411/User:Team7}

Their website can be found at
\\\url{http://green.smu.edu.sg/gspm2013/team07/}
\newpage