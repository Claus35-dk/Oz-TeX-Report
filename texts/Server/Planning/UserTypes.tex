\subsubsection{User types}
\label{s_actor-goal-list}
Below we give an overview of the functionality the server should provide to each dynamically defined user type that we have chosen for this version of the server.

\begin{description}
	\item [Non-registered user] \hfill \\
		This is the user type with the least options, only being able to browse the products and register to become a customer.
	\item [Customer]  \hfill \\
		The costumer is quite possibly the most common user type the service will have. The big difference between a customer and a non-registered user is the customer's ability to:
		\begin{itemize}
			\item buy/rent products
			\item buy credits
			\item download/view products
			\item rate products
		\end{itemize}
	\item [Content provider] \hfill \\
		The content provider is, as the name suggest, the kind of user who will be filling the service with products. They cannot rent, buy, or rate products, only upload and manage their own products. \todo{not true}{Claus}
	\item [Admin] \hfill \\
		%The admin is the most powerful user type with the power to administrate any registered user or product, including changing any piece of associated information.
		The admin is the god-like user type of the system with ultimate powers to alter the reality of registered users and products. With power over life and death, he may introduce new mortals and others gods to the butt-sex \todo{Michael, du bør nok fjerne det her igen}{Krüger} system and at any point of time exclude any sinner from heaven by banning them. He is able to see into the very account of any user, reading the most private information entered by a user (minus their password). You cannot see him any place, but through invisible, do not be tempted to believe that he do not exist, cause if you mess with him, he will kick your ass out of the system, literally! (See tex comment for original text. -Philip :)
\end{description}