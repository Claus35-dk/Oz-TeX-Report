\section{Introduction to collaboration with SMU}
In this project we had the amazing opportunity to collaborate with a group of students from the Singapore Management University.

The collaboration was started with a short video chat in order to say hello and get a first impression. After this we held two video chat meetings each week for the rest of the project.

In the beginning both parties made sure to formulate sentences very clearly and straight forward, trying not to say something that could be perceived with a different meaning than intended. We found that communicating via text was way easier as the accent was not a factor.
With support from text, every meeting became easier and more loose, and the conversation became natural with their best English speaker.

From our group we chose Claus as our liaison and they chose Lynette. They were the ones mainly responsible for communication between the groups and for keeping their own group well informed.

We were aware of the fact that our Singaporean group was of a different culture and that they might interpret something differently than we would. Fortunately such situations never occurred.

Singapore is seven hours ahead of Denmark during daylight saving time and eight hours when not. We usually held our meetings around noon Danish time, which is late afternoon Singaporean time. It was not hard to arrange meetings, but the time difference often made for long response times.

By the end of our collaboration each member from each group reflected upon the whole process. The reflections can be found on our shared wiki by using the URL provided below and in our appendix \textit{[INSERT APPENDIX REF]}.
\\\url{wiki.smu.edu.sg/is411/User:Team7}\\\\

Their website can be found at:
\\\url{http://green.smu.edu.sg/gspm2013/team07/}
\newpage