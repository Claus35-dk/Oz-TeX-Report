% !TeX spellcheck = en_US
\subsubsection{Web Service API}
At the beginning of the process we appointed one person to write a comprehensive API documentation for our web services.
The API should be able to handle all of the requirements agreed with SMU.

The document was made available to SMU via Google Drive. From here the SMU was able to add comments and ask questions about certain requests and their responses.

When parts of the API where approved by both ourselves and SMU the part was ready to be implemented as a web service.

This document would also become a place we regularly looked at when developing the web service itself, since the API dictated every possible outcomes for every request supported by the web service.

\subsubsubsection{How to read}
Each request is documented in the API documentation under the URI they use. Following the URI is a description of the requests to this URI, a HTTP header template and a list of whom is able to perform the requests.

After this initial information, the actual format of the request follows. They are categorised with up to four kinds of requests that can be made on (almost) every URI: GET, PUT, POST, and DELETE. \\
After each format, a description including request examples follows. \\
At the end of the examples a list of possible responses follows with explanation of what each response means.

For the complete API, see \apref{API_PDF}.