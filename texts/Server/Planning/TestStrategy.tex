\subsubsection{Test Strategy}
When we started coding the server we did not have any idea of what or how we wanted to test, which made debugging very difficult.
We used a lot of time on following calls through the different layers of the server to discover where an error or exception occurred.
A few tests where made after most of the debugging had been done to prove that the server performed as wanted.
That meant that we did not have a test strategy for the server.

A different approach to our solution could be that we in advance decided which parts of the server that we had to test and which parts where redundant to test. The level at which the tests should be made is the next to decide. When you make the tests, before or afterwards the code to be tested, is part of a test strategy as well.

The reason why we did not have any test strategy for the server, was not a conscious choice. We did not consider testing until most of the code had been written and debugged. Since we were lacking planning for the development we did not realize we had forgotten the testing. Not testing cost us more time than any test writing since the debugging was near impossible and very time consuming. By testing the individual parts of the server, we would have a much better idea of what kind of exception would be thrown from which part of the server. That in turn would make debugging much easier, since we would know where a exception would come from and we would most likely know why an exception is thrown. Using functions or types wrong would be discovered faster if we had tested the server more thoroughly, again since we would know it from the responses from the individual parts of the server.
