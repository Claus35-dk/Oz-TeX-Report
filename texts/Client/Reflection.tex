\subsection{Reflection}
Our work on the client has been focused and well executed. We planned to build the client in a cool widget-composite structure. We wrote a lot of structural code before finally being able to see the result of our hard work. It turned out even better than we had hoped. Whenever we wanted to add functionality it was always painless and much quicker to implement than one would think. If you were to add a new widget you should have knowledge of HTML, but you will rarely write any actual HTML. You add attributes to the widget and the HTML is generated for you. This also has the benefit of consistent code since it is generated alike for all widgets.

\todo{Remove the following?}{Claus}
A minor fault in our structure often became apparent whenever we had an error somewhere in our code. Often the page would render blank and our server error log would complain about an error in a ToHtml-method, but not in which subclass of Widget the error originated. Usually this was no big problem since we knew which changes we had made since the last time it worked.

The work on the client was well structured. We used a backlog with all the items (mostly widgets) related to the client and report. Each item was described and estimated in hours. When a group member was done with one item he assigned himself to the next. The backlog gave us an overview of which items needed to be done, which was in progress, which was done and how many hours of work lay ahead of us. We used \url{www.Scrumwise.com} for keeping this backlog.
\newpage