\subsubsection{Communication}
For our collaboration we have used five technologies to communicate and share informations; three for communication and two for sharing information between the groups.

For sharing information we have used Dropbox and Google Drive. \vspace{-5mm}
\begin{description}
	\item Dropbox \\
		We chose to use Dropbox for sharing non-document files. We chose Dropbox since it is a program all of us already had installed, used before and are a very easy tool for sharing files which multiple people are not to change at the same time.
	\item Google Drive \\
		For files that multiple people might edit at the same time, and document files in general, we chose Google Drive, which allows for multiple people tp edit the same file at the same time. Since everyone already had a Google Drive account it was easy to set up.
		Google Drive was mainly used for sharing the API for the web service.
\end{description}

When commuting with SMU we have used Skype, Google Hangout and email.
\vspace{-5mm}
\begin{description}
	\item Skype \\
		For most of our meetings with the SMU we have used Skype video calls. Skype was a great improvement over the video conference tool used by ITU.
		Skype was also used as an instant message service, when small problems needed to be solved and questions answered. \\
		The drawback of using Skype as a video conference tool, is that when using the free plan (which we did), only two Skype accounts can be in a video conference at a time.
	\item Google Hangout \\
		If the group members of one of the groups where not able to meet at one place to conduct a video conference we used Google Hangout. With this tool up to 8 individual accounts can participate in one video conference at a time.
	\item Email \\
		To inform and communicate beside the video conferences the appointed liaisons email each other and passed the information on to the rest of their group.
		The use of email eliminated problems that might arise by different and unknown dialects.
\end{description}