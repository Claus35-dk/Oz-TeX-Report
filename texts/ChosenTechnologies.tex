\section{Chosen technologies}
In the very beginning, even before our collaboration with SMU we talked about which technologies we would like to use. Later, we had the parts concerning SMU approved by them.

We chose to use F\Sh for developing our server. We had all recently been introduced to the programming language, and were very exited about it. We liked the completely different idea on which F\Sh was based. We all attended a course in the language, but wanted to use it in a bigger project, to put it into perspective. It was therefore decided unanimously that we wanted to use that language for our server. The choice was approved by our lecturer.

For SMU to be able to use our server we had to expose some web services. Two different technologies were at hand. Those were SOAP and REST. We chose to use REST because is much simpler than using SOAP, and this made it less time consuming to implement in the client. We had anticipated right from the start that there would not be that much time left to implement our client. 

We faced some difficulties in exposing web services as we chose to use F\Sh for programming. We learned that webservices in F\Sh was not at a state that was easy for us to use, and therefore we planned to implement a thin service-layer in C\Sh. This should then call the corresponding parts of our F\Sh code. This, however, turned out to not be so thin as we planned.

For storing data we all agreed to use the SQL format. We all used this before with great success and wanted to keep using it. After this choice was made we had to choose between two different implementations. MySQL and MSSQL. We started out by agreeing on using MySQL, as we all had worked with it before, and favored it to MSSQL. We also talked to our lecturer about this and he was indifferent about it. However, when we started making our data model, and deployed it on the server it became apparent that it was easier to use MSSQL, as we did not need to setup server in any way.

Lastly we decided on which technology we wanted to write our own client in. We did not want to program our client in ASP, or any of the other Microsoft web languages. Some of us has had some experience with PHP, and one in the group had extensive experience in it. We decided that we wanted to try programming our client in PHP. We asked our lecturer, and it was approved.
\newpage