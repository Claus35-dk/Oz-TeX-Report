\subsubsection{Web Service API}
At the beginning of the process we appointed a single person to write a comprehensive API documentation for our web services.
The API should be able to handle all of the requirements agreed with SMU.

The document was made available to SMU via Google Drive. From here the SMU group was able to add comments and ask questions about requests and their responses.

When parts of the API were approved by both ourselves and SMU the part was ready to be implemented as a web service.

Being as comprehensive as it is not only served as a users manual for SMU, but also as an implementation guide for us, since the API dictated every possible outcome for every request supported by each web service. After implementation it was additionally a perfect means of determining whether the implemented functionality was complete.

\subsubsubsection{How to read}
Each service is documented in the API documentation under the URI it uses. Following the URI is a description of the requests to this URI, a HTTP header template and a list of whom is able to perform the requests.

After this initial information, the actual format of the request follows. They are categorised into up to four kinds of requests that can be made to (almost) every service: GET, PUT, POST, and DELETE. \\
After the name of each request type, a description including request examples follows. \\
At the end of the examples a list of possible responses follows with explanation of what each response means.

For the complete API, see \apref{API_PDF}.