\section{Target audience}
We tried to make our back-end server and data model as flexible and capable as possible. We wanted it work with whatever audience one might choose to target with ones client. We chose to do so, though it is time consuming, because we did not want it set any limits to any client which wanted to use our service. We also had in mind that both we and our Singaporean group was going to use the service and we did not know what sort of site they had in mind.

For our client we really wanted to do something like Netflix, but without the limitations of only video content. 
\\We see our target content providers as companies who cheaply wants to make content available for the masses and we want to make it up to the providers to set the price and to decided how much money they want to earn, if any at all.
\\To make our client attractive for content providers we wanted to support both free products, paid products as well as products for rent. 

We see our client customers as internet familiar. We want our client to be easy to use and for customers at any age. A kids version would be nice to have, but it is not something we will look into at this time.

We expect to make earnings from commercials and by feeing content providers per sale. But one should keep in mind that our service relies heavily on its content and its content providers, as many similar services does and our competition is strong.

Our very wide choice of target audience might prove not to be lucrative. Netflix is a very specific service and people know exactly what they can get from that service. Our wide array of services does not limit us, per se, but a more simple service can focus stronger on its target and there by hit more accurately.
\newpage